A crucial development in the history of CA research was the proof
\cite{cook2004universality} that certain CA rules are Turing complete
(in particular, Rule 110 in Wolfram's numbering system
\cite{wolfram1994cellular} enjoys this property).

Earlier classic works
\cite{langton1990computation,mitchell1993revisiting,packard1988adaptation}
exploring related ``edge of chaos'' effects.  In
\cite{packard1988adaptation,mitchell1993revisiting,mitchell1994evolving},
genetic algorithms are used to search the space of CA rules via
crossover and mutation.  This sort of evolution is global and is
connected with the CA rule by a derived parameter, ``Langton's
$\lambda$'' (cf. \cite{langton1990computation}).  An overview of the
``EvCA'' programme is presented in \cite{hordijk2013evca}.

Closest to the work presented here is \cite{sipper1997evolution},
which introduces the paradigm of \emph{cellular
  programming}.\footnote{``\emph{As opposed to the standard genetic
    algorithm, where a population of independent problem solutions
    globally evolves, our approach involves a grid of rules that
    coevolves locally}'' \cite[p. 74]{sipper1997evolution}.}  As the
name indicates, this approach is a fusion of cellular automata and
genetic programming ideas.  Local evolution of the CA rule makes use
of a local ``fitness'' (\cite[pp. 79--81]{sipper1997evolution}), as
the systems are evolved to perform certain global computational tasks.

In the current effort, although we are interested in behaviour that
tends towards edge-of-chaos effects, system evolution is not directly
guided by a specific fitness criterion, but only by variations on the
``crossover'' mechanism.

One early application of cellular programming is to evolutionary game
theory, a field to which it bears a strong resemblance
(e.g.~\cite{nowak1992evolutionary}).  We do not consider game
theoretic approaches in this work, despite being inspired by the
social metaphors that are involved (e.g.~\cite{nowak2006five}).  In
our thinking we often switch between conceptual/symbolic,
social/ethical, biological/genetic, and physical/geometric metaphors.

In this connection it is worth mentioning some recent work
\cite{goerg2012licors,goerg2012mixed} that continues in the earlier
tradition of the EvCA project (cf. \cite{hordijk2001upper}), making
use of a relativistic ``light cone'' analysis to identify structure in
CAs.  The current paper does not pursue any detailed \emph{post hoc}
analysis of CA behaviour, although we plan to explore this further in
subsequent work.  Finally, although not focusing on CAs per se,
\cite{hofstadter1995prolegomena} outlines a set of criteria for the
design of systems that exhibit ``emergent'' intelligence which helped
to motivate the present effort.


